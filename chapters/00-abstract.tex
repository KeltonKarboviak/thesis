% !TEX root =  ../main.tex

	This work details the development of the \toolname\ (\toolnameshort), a tool for analyzing and grading the quality of Approach and Landing phases of flight for the National General Aviation Flight Information Database (NGAFID).  General Aviation (GA) accounts for the highest accident rates in Civil Aviation, and the Approach and Landing phases are when a majority of these accidents occur.  Since GA aircraft typically lack most of the sophisticated technology that exists within Commercial Aviation, detecting phases of flight can be difficult.  Moreover, because of the high variability in GA operations and abilities of the pilot, detecting unsafe flight practices is also not trivial.  This thesis details the usefulness of an event-driven approach in analyzing the quality and risk level of an Approach and Landing.  In particular, the application uses several parameters from a flight data recorder (FDR) to detect the phases of flight, detect any safety exceedances during the phases, and assign a metrics-based grade based on the accrued number of risk levels.  The goal of this work is to improve the post-flight debriefing process for student pilots and Certified Flight Instructors (CFI) by augmenting the currently limited feedback with metrics and visualizations.  By improving the feedback available to students, it is believed that it will help to correct unsafe flying habits quicker, which will also help reduce the GA accident rates in the long-term.  The data was collected from a Garmin G1000 FDR glass cockpit display on a Cessna C172 fleet.  The developed application is able to successfully detect go-arounds, touch-and-goes, and full-stop landings as either stable or unstable with an accuracy of 98.16\%.  The \toolnameshort\ can be used to provide post-flight statistics and user-friendly graphs for educational purposes.  It is capable of assisting both new and experienced pilots for the safety of themselves, their organization, and GA as a whole.