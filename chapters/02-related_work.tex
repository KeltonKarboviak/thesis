% !TEX root =  ../main.tex

\chapter{Related Work} \label{ch:related_work}

%----------------------------------------
% AIRCRAFT OPERATIONS
%----------------------------------------
\section{Aircraft Operations}

	In Dr. Ed Wischmeyer's paper, \emph{The Myth of the Unstable Approach}~\cite{wischmeyer2004the-myth}, he discusses how the term ``unstable approach'' is now becoming too vague to be used in accident and incident reports.  He argues there are too many factors that play into an approach; therefore, labeling it solely as an ``unstable approach'' is not sufficient.  This aligns with one of the goals of the \toolname\ in that it was developed to detect unstable approaches and be able to state what the specific parameter was that caused the approach to be unstable.  In doing this, it allows for finer-grained statistics to be generated, which can reveal further patterns to be detected within an organization if it becomes a wide-spread problem.
    
    Nazeri \etal~\cite{Nazeri:2008:Analyzing-Relat} researched accident and incident data from several different commercial flight data sources in order to discover the factors that cause those events.  They created eight high-level categories, each with sub-factors, for classification.  They used an algorithm to analyze the data for correlations between different attribute-value pairs across the accident and incident data sets.  A factor support ratio was calculated for each attribute-value pair and ranked in decreasing order to find the most significant factors.  The following high-level factors were the four top ranked in order:  company, air traffic control, pilot, and aircraft.  They also did a time-series analysis of the data for the ten-year period in which the data was collected (1995-2004).  This time-series data showed the pilot and aircraft factors are generally decreasing over time, while the air traffic control factors are generally increasing.  By uncovering these patterns and analyzing them over time, they were able to find the factors that are leading causes for accidents/incidents and can address these factors for improvement.


%----------------------------------------
% POST-FLIGHT EVALUATION TOOLS
%----------------------------------------
\section{Post-Flight Evaluation Tools}

	There are several software projects that have created post-flight evaluation tools, which a pilot can use to analyze their performance during various phases of flight.  Each project has a similar methodology, but varying presentation techniques.  Knighton and Claramunt~\cite{knighton2001an-aeronautical} created a system that included hardware for collecting real-time flight data and an interactive graphical user interface (GUI).  The GUI is capable of 2D flight re-animation and simple plots of the aircraft's vertical profile throughout the flight.  The flight re-animation also has a panel with indicators showing the real-time sensor data at each time step.  They found through experiments, using volunteers, that their interface was intuitive and encouraged exploration of different aspects of flight performance.  Despite the usefulness of the flight parameter graphs, the downside to their research is that any analysis has to be performed manually by the user as there are no automated analysis results provided.
    
    Masiulionis and Stankunas~\cite{masiulionis2017review} provide a review of several software packages for flight analysis including \textit{IGC Flight Replay}, \textit{OziExplorer}, \textit{GPS TrackMaker}, and \textit{ArcGIS}.  They found that none of the packages provided all the aspects they sought:  interactively enabling/disabling map layers, graphing multiple flight parameters on a single plot, and high-resolution maps.  Thus, they experimented with flight analysis and visualization using \textit{Google Earth}.  \textit{Google Earth} met all of their standards, but the only disadvantage is that the altitude and speed graphs can become compressed for flights with an extended duration.  It does not include any features to resize or drill-down into the graphs to make them more viewable.
    
    Goblet \etal~\cite{goblet2016phase,goblet2015identifying} researched into automatically classifying phases of GA flights.  The focus is on the climb, cruise, and descent phases as they are the most difficult phases to identify in GA due to the variation of mission profiles and purposes of flight when compared to Commercial Aviation. Several methods were explored for identifying these phases:  altitude-based and smoothing-and-differentiation-based (which includes down-sampling, moving average, and local regression).  It was found the best method to use for a particular flight is determined by specific characteristics found within that flight.  An algorithm is then given to automatically select the best method for each flight.  Although the climb, cruise, and descent phases are the most difficult to identify, it is known that the approach, landing, and go-around phases are the most critical and dangerous in GA (as discussed in \Cref{ch:introduction}), thus is the reason this work focuses on analysis of those phases. 
    
    Fala and Marais~\cite{fala2016detecting} developed a method of detecting unsafe aircraft parameters, termed ``safety events'', in GA flights.  Similar to phase identification, detecting safety events in GA is difficult due to the variability in operations.  In their research, they only focus on the approach phase and provide the numerical parameter limits for a Cirrus SR20 aircraft during the approach phase.  For each parameter, they provide a Level 1 and Level 2 limit stating that Level 2 is more dangerous than Level 1.  They analyzed the approach phases from a sample of 23 flights using their defined thresholds and performed a one-way ANOVA analysis to evaluate whether the average number of instances for each type of safety event was similar.  They found their initial threshold definitions were not similar enough across the parameters.  They revised the definitions, re-analyzed the flights, and performed another ANOVA analysis to find that they were then significantly similar.  When detecting safety events, Fala and Marais did not distinguish between sporadic exceedances of the thresholds and a span of consecutive exceedances.  This differs from the work in this project, where an event can span multiple seconds instead of only single time steps.
	

%----------------------------------------
% NGAFID RELATED WORK
%----------------------------------------
\section{NGAFID Related Work}

	Other work on the NGAFID has focused in two different areas.  In the first, Desell \etal\ have examined methods based on the prediction of flight data parameters using recurrent neural networks (RNNs).  They have shown that training Jordan and Elman RNNs using evolutionary algorithms such as Particle Swarm Optimization and Differential Evolution can provide strong predictive results for flight data parameters such as airspeed, altitude, pitch, and roll~\cite{desell2014evolving}; and that these results can be further refined utilizing a novel neuroevolution technique based on ant colony optimization to evolve the structure of the RNNs~\cite{desell2015evolving}. Further work by ElSaid \etal\ has utilized Long Short-Term Memory (LSTM) RNNs to predict aircraft engine vibration events~\cite{elsaid2016vibration,elsaid2016thesis}. \note{Insert new paper from GECCO}
    
    The other area has been in utilizing unsupervised machine learning methods to detect anomalous flights.  Clachar \etal\ have used self organizing maps (SOMs), a type of neural network, to both cluster time series flight data and identify anomalous flights~\cite{sophine2014identifying,sophine2016phd} based on approach phases. The SOMs provided significant benefits in terms of parallelism and performance over other clustering methods such as DBSCAN~\cite{ester1996a-density-based}.


%----------------------------------------
% DATA MINING TECHNIQUES
%----------------------------------------
\section{Data Mining Techniques}

	Harris \etal~\cite{harris-jr.1998recent} of MITRE Corporation mined accident and incident reports provided by the International Civil Aviation Organization (ICAO) in order to determine the specific attributes that were the cause in each kind of report and also needed to be considered ``interesting'' (i.e., anything that is an exception to commonly accepted knowledge among aviation experts) by the aviation expert who collaborated with them.  They discovered that using traditional data mining methods was not sufficient enough to find ``interesting'' results.  This is because experts have studied the field of aviation so extensively that all the obvious rules and correlations have already been discovered.  Next, they developed their own system called Smithers, which uses a technique called attribute focusing, that finally uncovered an interesting correlation which shows that having an advanced heads up display (HUD) can help reduce the amount of damage as a result of a runway incursion\footnote{Defined by the Federal Aviation Administration (FAA) as, ``any occurrence at an aerodome involving the incorrect presence of an aircraft, vehicle, or person on the protected area of a surface designated for the landing and take off of aircraft''~\cite{federal-aviation-administrationrunway}.}.  This shows that even though aviation safety has been studied extensively, there are still new correlations that can be made when applying several different data mining methods.
    
    Matthews \etal~\cite{matthews2013discovering} performed similar research in which their goal was to find anomalous data in flights.  They differ in the fact that they used algorithms that could analyze at both a fleet-level and flight-level.  Doing this allowed them to find anomalies for an entire organization or just a single flight, which makes it very useful in order to find patterns of problems.  This idea is similar to the NGAFID project in which flight data can be analyzed on multiple levels while giving statistics for each.
